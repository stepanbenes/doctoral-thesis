% Text of abstract should have three parts: Motivation. This paper. Summary.
\noindent
Finite element analysis is a process of modeling physical reality that consists of several phases -- model generation, meshing, attribute assignment, solution, and post-processing. With the ongoing desire to solve more complex systems with better and better precision, an analysis has to process enormous amount of data in each of its phases. Traditional unstructured file-based representation of the mesh, input parameters, and the results from the solution is the bottleneck of the entire process. It lacks the scalability and complicates the development of the tools for engineers and researchers that are either preparing the input to FEM, or interpreting the output from FEM.

Limitations of the standard file-based approach are the motivation to re-think the entire process of data management in FEA. The focus of the thesis is mainly on the post-processing of the results and the way the results are stored, transfered, and visualized. However, the thesis describes the design and implementation of the complete FEA data management system that connects all the parts of the finite element analysis, providing query interface and remote access over the Internet. There is proposed the new storage format for representation of FEM results that provides the persistent representation of visual filters to simplify the implementation of a post-processor.

The main feature of the storage format is the support for compression of FEM results. The compression method based on singular value decomposition is proposed. The method is able to compress arbitrary results from FEM using low-rank approximation matrices. The compression ratio is at most 10\% for all tested results. In many cases, the compression ratio is bellow 1\% of the original size, while the relative approximation error is kept under $10^{-5}$.

To demonstrate the proposed methods, the thesis describes the implementation of two post-processors. The desktop post-processor is a feature-rich visualization tool that allows to visualize the data in various formats including the new proposed storage format. It is able to create efficient surface representation of an arbitrary finite element mesh and it implements advanced techniques for manipulation with the mesh entities. The web-based post-processor is a simple cross-platform application that demonstrates the benefits of the proposed storage format. It is able to visualize the simulation results located in a remote storage. As the hard work connected with processing of the results is offloaded to a server, the web application is just a thin client that works even on devices with limited CPU and memory resources.\\
