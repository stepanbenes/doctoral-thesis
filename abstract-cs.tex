\noindent
Konečně prvková analýza je proces sloužící k simulaci průběhů fyzikálních veličin, který sestává z několika fází -- vytvoření geometrického modelu, generování sítě konečných prvků, přiřazení parametrů modelu, konečně prvkový výpočet a zpracování výsledků. S pokračující snahou o stále vyšší přesnost výpočtu, každá z fází analýzy musí zpracovávat obrovské množství dat. Tradiční reprezentace sítě, vstupních parametrů a výsledků založená na obyčejných nestrukturovaných souborech je úzkým hrdlem celého procesu. Tento fakt komplikuje vývoj nástrojů pro inženýry a vědce, kteří připravují vstupní data do MKP nebo interpretují výsledky z MKP.

Nevýhody a omezení tradičního přístupu založeného na souborech je motivací pro výrazné přepracování celého způsobu nakládání s daty v konečně prvkové analýze. Práce je zaměřena především na zpracování výsledků z MKP a na způsob jejich ukládání, přenos a zobrazování. Nicméně, dizertační práce popisuje také návrh a implementaci kompletního systému pro správu dat, který propojuje všechny části konečně prvkové analýzy a poskytuje rozhraní pro dotazování nad daty a vzdálený přístup přes Internet. Je zde rovněž představen nový formát pro reprezentaci výsled\-ků z MKP, který mimo jiné podporuje uložení vizuálních filtrů aplikovaných na data, což usnadňuje implementaci post-procesoru.

Hlavní výhoda nového formátu je podpora pro kompresi dat. Kompresní metoda založená na singulárním rozkladu je představena a popsána. Metoda je schopna zkompresovat libovolnou sadu výsledků z MKP použitím aproximace maticí s nižší hodností. Kompresní poměr je nejvýše 10\% pro všechny testované výsledky. V~mno-ha případech je kompresní poměr pod 1\% původní velikosti, zatímco relativní chybu aproximace se podařilo udržet pod $10^{-5}$.

Pro demostraci uvedených metod dizertační práce rovněž popisuje implementaci dvou post-procesorů. Desktopový post-procesor je vizualizační nástroj, který umožňuje zobrazovat data v různých formátech včetně nově navrženého formátu podporujícího kompresi výsledků z MKP. Post-procesor je schopen vytvořit efektivní reprezentaci konečně prvkové sítě a implementuje pokročilé techniky pro manipulaci s uzly, hranami a prvky sítě. Webový post-procesor je jednoduchá multi-platformní aplikace, která demonstruje výhody nového formátu. Umožňuje zobrazit výsledky z MKP umístěné ve vzdáleném úložišti. Díky tomu, že výpočetně náročné operace související se zpracováním výsledků jsou prováděny na vzdáleném serveru, webová aplikace je pouze tenký klient, který je schopen pracovat i na zařízeních s velmi omezeným výkonem a pamětí.\\
