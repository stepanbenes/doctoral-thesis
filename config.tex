%%%%%%%%%%%%%%%%%%%%%%%%%%%%%%%%%%%%%%%%%
% Masters/Doctoral Thesis 
% LaTeX Template
% Version 2.4 (22/11/16)
%
% This template has been downloaded from:
% http://www.LaTeXTemplates.com
%
% Version 2.x major modifications by:
% Vel (vel@latextemplates.com)
%
% This template is based on a template by:
% Steve Gunn (http://users.ecs.soton.ac.uk/srg/softwaretools/document/templates/)
% Sunil Patel (http://www.sunilpatel.co.uk/thesis-template/)
%
% Template license:
% CC BY-NC-SA 3.0 (http://creativecommons.org/licenses/by-nc-sa/3.0/)
%
%%%%%%%%%%%%%%%%%%%%%%%%%%%%%%%%%%%%%%%%%

%----------------------------------------------------------------------------------------
%	PACKAGES AND OTHER DOCUMENT CONFIGURATIONS
%----------------------------------------------------------------------------------------

\documentclass[
    11pt, % The default document font size, options: 10pt, 11pt, 12pt
    %oneside, % Two side (alternating margins) for binding by default, uncomment to switch to one side
    english, % ngerman for German
    singlespacing, % Single line spacing, alternatives: onehalfspacing or doublespacing
    %draft, % Uncomment to enable draft mode (no pictures, no links, overfull hboxes indicated)
    %nolistspacing, % If the document is onehalfspacing or doublespacing, uncomment this to set spacing in lists to single
    %liststotoc, % Uncomment to add the list of figures/tables/etc to the table of contents
    %toctotoc, % Uncomment to add the main table of contents to the table of contents
    %parskip, % Uncomment to add space between paragraphs
    %nohyperref, % Uncomment to not load the hyperref package
    headsepline, % Uncomment to get a line under the header
    %chapterinoneline, % Uncomment to place the chapter title next to the number on one line
    %consistentlayout, % Uncomment to change the layout of the declaration, abstract and acknowledgements pages to match the default layout
    ]{MastersDoctoralThesis} % The class file specifying the document structure
    
\usepackage[utf8]{inputenc} % Required for inputting international characters
\usepackage[T1]{fontenc} % Output font encoding for international characters

\usepackage{palatino} % Use the Palatino font by default

\usepackage[backend=bibtex,style=numeric,giveninits=true,natbib=true,sorting=none]{biblatex} % Use authoryear citation style for "Author, year" citation style

\addbibresource{references.bib} % The filename of the bibliography

%\usepackage{refcheck} % checks for correct reference usage and produces warnings; enable \nocite command

\usepackage[autostyle=true]{csquotes} % Required to generate language-dependent quotes in the bibliography

\usepackage{float} % In order to use the [H] option for figure alignment

\usepackage{fancyvrb} % keep tabs in verbatim sections, see: https://tex.stackexchange.com/questions/231083/how-to-stop-verbatim-from-converting-tabs-to-spaces

\usepackage{listings} % enable to reference source code, see: ttps://tex.stackexchange.com/questions/74155/how-to-treat-verbatim-as-a-block-that-can-be-referenced-just-like-a-figure

\usepackage{amssymb} % The amssymb package provides various useful mathematical symbols

\usepackage{mathtools} % math tools (ceiling signs, see: http://tex.stackexchange.com/questions/42271/floor-and-ceiling-functions)
\DeclarePairedDelimiter{\ceil}{\lceil}{\rceil}

% following packages allows to write algorithms as pseudocode
\usepackage{amsmath}
\usepackage{algorithmicx}
\usepackage{algorithm}
\usepackage{algpseudocode}
\algnewcommand\algorithmicinput{\textbf{INPUT:}}
\algnewcommand\INPUT{\item[\algorithmicinput]}
\algnewcommand\algorithmicoutput{\textbf{OUTPUT:}}
\algnewcommand\OUTPUT{\item[\algorithmicoutput]}

\usepackage{caption}
\captionsetup{width=0.8\textwidth} % global width of captions in figures, tables, etc. https://tex.stackexchange.com/questions/110393/too-wide-figure-caption

%----------------------------------------------------------------------------------------
%	COMMAND DEFINITIONS
%----------------------------------------------------------------------------------------

\definecolor{todoColor}{rgb}{1,0.55,0.2}
\newcommand{\todo}[1]{\textit{\color{todoColor}#1}}

% define commands to keep the formatting separated from the content 
\newcommand{\keyword}[1]{\textbf{#1}}
%\newcommand{\tabhead}[1]{\textbf{#1}}
\newcommand{\code}[1]{\texttt{#1}}
\newcommand{\file}[1]{\texttt{\bfseries#1}}
\newcommand{\option}[1]{\texttt{\itshape#1}}
\newcommand{\tabhead}[1]{\textbf{#1}}
\newcommand*\diff{\mathop{}\!\mathrm{d}} % dx
\newcommand*\Diff[1]{\mathop{}\!\mathrm{d^#1}} % d^2x
\newcommand{\mtrx}[1]{\mathbf{#1}}

%----------------------------------------------------------------------------------------
%	MARGIN SETTINGS
%----------------------------------------------------------------------------------------

% WRONG VERSION:
%\geometry{
%    paper=a4paper, % Change to letterpaper for US letter
%    inner=2.5cm, % Inner margin
%    outer=3.8cm, % Outer margin
%    bindingoffset=.5cm, % Binding offset
%    top=1.5cm, % Top margin
%    bottom=1.5cm, % Bottom margin
%    %showframe, % Uncomment to show how the type block is set on the page
%}

% E-VERSION:
\geometry{
    paper=a4paper, % Change to letterpaper for US letter
    inner=3.4cm, % Inner margin
    outer=3.4cm, % Outer margin
    bindingoffset=0cm, % Binding offset
    top=1.5cm, % Top margin
    bottom=1.5cm, % Bottom margin
    %showframe, % Uncomment to show how the type block is set on the page
}

% PRINT VERSION:
%\geometry{
%    paper=a4paper, % Change to letterpaper for US letter
%    inner=3.3cm, % Inner margin
%    outer=3.0cm, % Outer margin
%    bindingoffset=0.5cm, % Binding offset
%    top=1.5cm, % Top margin
%    bottom=1.5cm, % Bottom margin
%    %showframe, % Uncomment to show how the type block is set on the page
%}

%----------------------------------------------------------------------------------------
%	THESIS INFORMATION
%----------------------------------------------------------------------------------------
\title{Doctoral Thesis}
\thesistitle[0, 0]{Multimesh Methods for Data Visualization and Finite Element Analysis} % Your thesis title, this is used in the title and abstract, print it elsewhere with ttitle
\thesistitlecs{V\'ices\'i\v{t}ov\'e metody pro vizualizaci dat a v\'ypo\v{c}ty MKP} % Your thesis title in Czech, this is used in the title and abstract, print it elsewhere ith \ttitlecs
\supervisor{prof. Ing. Jaroslav Kruis, Ph.D.} % Your supervisor's name, this is used in the title page, print it elsewhere with \supname
\examiner{} % Your examiner's name, this is not currently used anywhere in the template, print it elsewhere with \examname
\degree{Doctor of Philosophy} % Your degree name, this is used in the title page and abstract, print it elsewhere with \degreename
\author{Ing. \v{S}t\v{e}p\'{a}n Bene\v{s}} % Your name, this is used in the title page, print it elsewhere with \authorname
\authornodegree{\v{S}t\v{e}p\'{a}n Bene\v{s}} % Your name without degrees, this is used in the abstract and declaration page, print it elsewhere with \authornamenodegree
\addresses{Th\'akurova 7, 166 29 Praha 6} % Your address, this is not currently used anywhere in the template, print it elsewhere with \addressname

\subject{Engineering software} % Your subject area, this is not currently used anywhere in the template, print it elsewhere with \subjectname
\keywords{Finite Element Method (FEM), Finite Element Analysis (FEA), Post-processing, Data visualization, Data compression, Data management, Singular Value Decomposition SVD)} % Keywords for your thesis, this is not currently used anywhere in the template, print it elsewhere with \keywordnames
\keywordscs{Metoda kone\v{c}n\'ych prvk\r{u} (MKP), Kone\v{c}n\v{e} prvkov\'a anal\'yza, Vizualizace dat, Komprese dat, Spr\'ava dat, Singul\'arn\'i rozklad (SVD)} % Keywords or your thesis, this is not currently used anywhere in the template, print it elsewhere with \keywordnames
\university{\href{http://www.cvut.cz}{Czech Technical University in Prague}} % Your university's name and URL, this is used in the title page and abstract, print it elsewhere ith \univname
\department{\href{http://mech.fsv.cvut.cz}{Department of Mechanics}} % Your department's name and URL, this is used in the title page and abstract, print it elsewhere with deptname
\faculty{\href{http://www.fsv.cvut.cz}{Faculty of Civil Engineering}} % Your faculty's name and URL, this is used in the title page and abstract, print it elsewhere with facname

\AtBeginDocument{
\hypersetup{pdftitle={Multimesh Methods for Data Visualization and Finite Element Analysis}} % Set the PDF's title to your title
\hypersetup{pdfauthor={\v{S}t\v{e}p\'{a}n Bene\v{s}}} % Set the PDF's author to your name
\hypersetup{pdfkeywords={Finite Element Method (FEM), Finite Element Analysis (FEA), Post-processing, Data visualization, Data compression, Data management, Singular Value ecomposition (SVD)}} % Set the PDF's keywords to your keywords
}

%----------------------------------------------------------------------------------------
%	SOURCE CODE STYLE
%----------------------------------------------------------------------------------------

\usepackage{inconsolata}

\definecolor{keywordsColor}{rgb}{0,0,0} % avoid coloring of keywords in comments
\definecolor{commentsColor}{rgb}{0,0,0} % avoid coloring of keywords in comments
\definecolor{stringsColor}{rgb}{0.64,0.08,0.08}
\definecolor{xmlcommentsColor}{rgb}{0.5,0.5,0.5}
\definecolor{typesColor}{rgb}{0.17,0.57,0.68}
\definecolor{numbersColor}{rgb}{0.25,0.6,0.83}

\usepackage{listings}
\lstset{language=[Sharp]C,
%captionpos=b,
%numbers=left, %Nummerierung
%numberstyle=\tiny, % kleine Zeilennummern
frame=lines, % Oberhalb und unterhalb des Listings ist eine Linie
showspaces=false,
showtabs=false,
breaklines=true,
showstringspaces=false,
breakatwhitespace=true,
escapeinside={(*@}{@*)},
commentstyle=\color{commentsColor},
morekeywords={partial, var, value, get, set},
keywordstyle=\color{keywordsColor},
stringstyle=\color{stringsColor},
basicstyle=\ttfamily\tiny,
}

\newcommand\JSONnumbervaluestyle{\color{numbersColor}}
\newcommand\JSONstringvaluestyle{\color{stringsColor}}

% switch used as state variable
\newif\ifcolonfoundonthisline

\makeatletter

\lstdefinestyle{json}
{
  showstringspaces    = false,
  keywords            = {false,true},
  alsoletter          = 0123456789.,
  morestring          = [s]{"}{"},
  stringstyle         = \ifcolonfoundonthisline\JSONstringvaluestyle\fi,
  MoreSelectCharTable =%
    \lst@DefSaveDef{`:}\colon@json{\processColon@json},
  basicstyle          = \ttfamily\tiny,
  keywordstyle        = \ttfamily\bfseries,
}

% flip the switch if a colon is found in Pmode
\newcommand\processColon@json{%
  \colon@json%
  \ifnum\lst@mode=\lst@Pmode%
    \global\colonfoundonthislinetrue%
  \fi
}

\lst@AddToHook{Output}{%
  \ifcolonfoundonthisline%
    \ifnum\lst@mode=\lst@Pmode%
      \def\lst@thestyle{\JSONnumbervaluestyle}%
    \fi
  \fi
  %override by keyword style if a keyword is detected!
  \lsthk@DetectKeywords% 
}

% reset the switch at the end of line
\lst@AddToHook{EOL}%
  {\global\colonfoundonthislinefalse}

\makeatother

\definecolor{positiveColor}{rgb}{0,1,0}
\definecolor{negativeColor}{rgb}{1,0,0}
\definecolor{neutralColor}{rgb}{0.6,0.6,0.6}

% https://tex.stackexchange.com/questions/12703/how-to-create-fixed-width-table-columns-with-text-raggedright-centered-raggedlef
\usepackage{array}
\newcolumntype{L}[1]{>{\raggedright\let\newline\\\arraybackslash\hspace{0pt}}m{#1}}
\newcolumntype{C}[1]{>{\centering\let\newline\\\arraybackslash\hspace{0pt}}m{#1}}
\newcolumntype{R}[1]{>{\raggedleft\let\newline\\\arraybackslash\hspace{0pt}}m{#1}}
