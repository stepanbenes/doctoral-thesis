\chapter{Related work}
\label{chapter:related-work}

\todo{TODO: State of the art and discussion about related work in the area of FEM results post-processing, storage formats, and cloud-based FEA \ldots}

\section{FEM post-processing / Mesh visualization}
% Vrstva je koncept, ktery se vyskytuje i u jinych postprocessoru (napr. Simscale)


\section{Storage of results / File formats}
% VTK format podporuje kompresi, ale bez znalosti obsahu dat. Pouze nejakou ZIP kompresi ci co. Neni to tak efektivni. Kazdy casovy krok ulozeny zvlast. Ale umi ukladat v ascii i binarne (base64 kodovani)

% pro vstupni geometrii:  STEP, IGES, BREP, STL, ... formats; 
% pro vysledky: VTK, GiD, ...

\cite{VTK2015}
\cite{GiDPostProcess}

\cite{Ivanyi2012}

% TODO: prozkoumat dalsi formaty pro ulozeni vysledku (vetsinou asi proprietarni, zadny standard neexistuje?)

\section{Compression methods}


\section{Web-based FEA}

% citovat papery
\cite{Ari2013}
\cite{Yu2010}
\cite{Peng2003}
\cite{Heber2007I}
\cite{Heber2007II}
\cite{Weng2011}
\cite{Chen2008}

% uvest Simscale
