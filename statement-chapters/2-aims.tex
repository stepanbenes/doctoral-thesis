\chapter{Aims}
\label{chapter:aims}

Here follows the list of aims that were set for the research work described in the thesis.

\begin{itemize}
    \item The main goal of the research work is to design a new storage format that will support compression of results from FEM. The format should have suitable properties that will allow it to be incorporated into a FEA data management system, possibly running on a remote server. There is understandable resistance against invention of new data formats in the area of information technology. A new format leads to fragmentation of user base and compatibility issues. Conversion tools need to be created and maintained. There should be a strong motivation for introduction of a new format. However, there is no standard format for representation of results from FEM. Each software package uses proprietary format with syntax suitable for its internal implementation. There is also lack of support for compression methods that fit the character of FEM results. Standard file-based format does not allow querying of specific information without the need to parse through the complete set of results.
    \item In addition, a suitable compression method need to be developed. Singular Value Decomposition (SVD) is the most promising method used for compression of FEM results in this research. Other methods that are investigated include Wavelet transform \cite{Li2014} and approximation of discrete values by continuous polynomial functions \cite{Benes2016, Benes2016Pollack}.
    \item Finally, the product of this research should be the implementation of two post-processors. The first is the standard desktop post-processor that will demonstrate the way of transition from the convential file-based formats to the proposed structured database format utilizing compression. The second post-processor is the web-based thin client intended to demonstrate the advantages of the proposed format when incorporated into a complex FEA running on a remote server.
\end{itemize}

% TODO: remove this section?
\section{Challenges}

The main challenge is the design of universal format that can hold the results from any FEM analysis. Results are composed of scalars, vectors, or tensors. Each field has different number of components. The results can be located in nodes or integration points. There may be a requirement to extrapolate the results from integration points to element nodes. There are various extrapolation strategies. The mesh can be different for each time step (e.g., in case of simulating the construction stages). The mesh can contain 1D, 2D, or 3D elements -- each of different type and approximation. The results from 3D simulations can be visualized on the surface of the mesh, in form of cross-sections or iso-areas, or as a vector field. The storage format should support efficient representation of all these forms of results.

Finite element solution and post-processing of results can be sometimes done on different computers. Complex FEA solution phase runs on a supercomputer or a performant cluster of workstations, but the results are post-processed on a common personal computer that has significantly less memory available. Typical personal computer has 8 to 32 GB of RAM, while the size of results can be in order of tens to hunderds of gigabytes. Also, the data to post-process have to be first transfered over the corporate network or the Internet. These conditions indicate the need for partitioning of data into smaller chunks and/or compression of the data.

The goal of compression methods is the significant reduction in size while preserving the quality (keeping the approximation error low). Unlike with image compression methods, where the main aspect is the human perception of the reconstructed image, the compression of FEM results should be able to guarantee the matematical acuracy of the approximations and the user should be able to specify a desired value of the approximation error. Another concern is the computational complexity of the compression algorithm. The compression will be performed only once after the solution phase is complete. The computational time should be an order of magnitude shorter compared to the solution phase. Decompression (reconstruction of the original data) should be very fast as it is supposed to be performed every time the data are post-processed on the end device, which can be ordinary PC or even mobile device. The ability to create animations should also be taken into account.

Other kind of challenge is to provide the data management system that will connect all the FEA phases, i.e., to provide links between the geometric model, the mesh entities, and the output values. A FEA project typically encompasses multiple simulations, each with different input or solver parameters. Multiple users are usually involved in the project and the system should help them to cooperate during the preparation of the input and allow to share the output of the analysis. All these aspects influence the design of the data management system.