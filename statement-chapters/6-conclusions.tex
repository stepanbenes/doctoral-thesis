\chapter{Conclusions}

There were three main goals of the thesis as described in Chapter \ref{chapter:aims}. The first goal was to design a new storage format for representation of the results from the finite element method with the support for compression. This goal was fulfilled. Besides the compression, the new storage format supports efficient querying of the data, unlike the standard unstructured file-based formats, which require parsing through the complete set of results to retrieve a specific information. Examples of the proposed format in form of JSON documents are given. Also, the application that can convert the results from the FEM solver to the new storage format was implemented. This converter application further supports generation of visual filters and it is designed to be run either locally on a PC or remotely in a cloud environment.

The second goal of the thesis was to investigate suitable methods for compression of results from FEM and develop a compression algorithm with reasonable performance characteristics and producing approximations with low and predictable error. The compression method based on singular value decomposition satisfies these requirements. The SVD compression method became the integral part of the storage format. The results of its application on real data have been presented. The algorithm is able to compress arbitrary data using low-rank approximation matrices. When the maximum allowed error was set to $10^{-5}$, the compression ratio was at most 10\% for all tested results. In many cases, the compression ratio can be even better -- bellow 1\% of the original size. The important property of the compression algorithm is the fact that the approximation error can be set in advance and there is a guarantee that it will not be exceeded. The disadvantage of the SVD-based compression method is the computational complexity. SVD is a very time-consuming operation. However, this operation is performed only once during the conversion of results from FEM solver to the storage format, before the post-processing is started. Also, the randomized version of the decomposition algorithm is much faster and can be used if a slight increase of the approximation error is tolerated. The detailed description of the SVD-based compression method is described in the thesis and also published in \cite{Benes2018}.

The third goal was the implementation of two post-processors that demonstrate the proposed methods. Both the desktop and the web post-processor were implemented and described in detail. The desktop post-processor is a feature-rich visualization tool that allows to visualize the data in various formats including the new proposed storage format. It is able to create efficient surface representation of an arbitrary finite element mesh and it implements advanced techniques for manipulation with the mesh entities. The detailed description of the implementation is presented in the thesis and published in \cite{Benes2015}. The web-based post-processor is a simple cross-platform application that is able to visualize the simulation results located in a remote storage. As the hard work connected with processing of the results is offloaded to the server, the web application is just a thin client that works even on devices with limited CPU and memory resources.

Besides the presented goals, the thesis also outlines the architecture of the data access system for complex FEA consisting of several independent services. The system is designed as a collaborative framework that can be accessed by users from different client devices. The web post-processor is built on top of this data management system, it directly communicates with the web API service to provide the user with the access to FEA simulations running on a remote server. The database schema for project and simulation related data is given as well as the description of individual services.

% Approximation by polynomials
As a not very suitable method for data reduction is considered the approximation of the FEM results by polynomial functions that is presented in the thesis and published in \cite{Benes2016} and \cite{Benes2016Pollack}. The method was inspired by the multigrid method (and generally other multi-mesh methods, hence the name of the thesis) that was at the beggining of the research work. The multigrid method allows to solve partial differential equations using the hierarchy of domain discretizations. The idea was to connect the FEM solution phase with the post-processing by reusing the mesh hierarchy used by the multigrid method also in the post-processing of the results. Although the presented approximation method is capable of a significant reduction of the data size (up to 2.5\% of the original size), the maximal approximation error can be very high (up to 100\% in extreme cases, e.g., when there are discontinuities or singulatities in the data). The unpredictibility of the error and the high decompression time are the reasons the method is excluded from the implementation of the post-processors in favor of the SVD compression method.
